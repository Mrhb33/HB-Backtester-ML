 PART 1: CORE INDICATORS & THEIR TRADING CONDITIONS

  1. RSI (Relative Strength Index)

  Standard Settings: 14-period (commonly used)

  Key Levels:
  ┌──────────┬────────────┬─────────────────────────────┐
  │  Level   │ Condition  │           Signal            │
  ├──────────┼────────────┼─────────────────────────────┤
  │ Above 70 │ Overbought │ Potential reversal downward │
  ├──────────┼────────────┼─────────────────────────────┤
  │ Below 30 │ Oversold   │ Potential reversal upward   │
  ├──────────┼────────────┼─────────────────────────────┤
  │ 30-70    │ Neutral    │ No extreme condition        │
  └──────────┴────────────┴─────────────────────────────┘
  LONG Entry Rules:
  1. RSI drops below 30 (oversold condition)
  2. Wait for RSI to cross back above 30
  3. Look for bullish confirmation:
    - Bullish candlestick patterns (hammer, engulfing)
    - Price rejection from support level
    - Volume increase on upward move

  SHORT Entry Rules:
  1. RSI rises above 70 (overbought condition)
  2. Wait for RSI to cross back below 70
  3. Look for bearish confirmation:
    - Bearish candlestick patterns (shooting star, bearish engulfing)
    - Price rejection from resistance level
    - Volume increase on downward move

  Exit Rules:
  - Take profit when RSI reaches opposite extreme level (30→70, 70→30)
  - Take partial profit at RSI 50 (midpoint)
  - Exit if RSI shows divergence (price makes higher high, RSI makes lower high)

  Divergence Strategy:
  - Bullish Divergence: Price makes lower low, RSI makes higher low → Reversal signal
  - Bearish Divergence: Price makes higher high, RSI makes lower high → Reversal signal

  Sources:
  - https://www.investopedia.com/articles/active-trading/042114/overbought-or-oversold-use-relative-strength-index-find-out.asp
  - https://www.oanda.com/us-en/trade-tap-blog/analysis/technical/mastering-rsi-trading-strategies/
  - https://www.quantifiedstrategies.com/70-30-rsi-trading-strategy/

  ---
  2. MACD (Moving Average Convergence Divergence)

  Standard Settings:
  - Fast EMA: 12-period
  - Slow EMA: 26-period
  - Signal Line: 9-period EMA of MACD

  Components:
  - MACD Line (Fast EMA - Slow EMA)
  - Signal Line (9-period EMA of MACD)
  - Histogram (MACD Line - Signal Line)

  Bullish Crossover (Buy Signal):
  - MACD line crosses ABOVE signal line
  - Stronger if both lines are below zero line (early trend)
  - Histogram moves above zero

  Bearish Crossover (Sell Signal):
  - MACD line crosses BELOW signal line
  - Stronger if both lines are above zero line (early trend)
  - Histogram moves below zero

  Histogram Divergence Strategy:
  - Bullish Divergence: Price makes lower low, Histogram makes higher low
  - Bearish Divergence: Price makes higher high, Histogram makes lower high

  Zero Line Context:
  - Both lines above zero = Bullish market mode
  - Both lines below zero = Bearish market mode

  Sources:
  - https://www.investopedia.com/articles/forex/05/macddiverge.asp
  - https://chartschool.stockcharts.com/table-of-contents/technical-indicators-and-overlays/technical-indicators/macd-histogram
  - https://www.oanda.com/us-en/learn/indicators-oscillators/determining-entry-and-exit-points-with-macd/

  ---
  3. Moving Average Crossover (Golden/Death Cross)

  Classic Configuration:
  - Short MA: 50-day SMA/EMA
  - Long MA: 200-day SMA/EMA

  Golden Cross (Bullish):
  - 50-day MA crosses ABOVE 200-day MA
  - Signals potential start of uptrend

  Death Cross (Bearish):
  - 50-day MA crosses BELOW 200-day MA
  - Signals potential start of downtrend

  Entry Rules (LONG):
  1. Wait for 50-day MA to cross above 200-day MA
  2. Mark current price action range
  3. Wait for pullback to MA zone
  4. Enter on lower timeframe confirmation
  5. Consider price above both MAs for additional confirmation

  Entry Rules (SHORT):
  1. Wait for 50-day MA to cross below 200-day MA
  2. Mark current price action range
  3. Wait for pullback to MA zone
  4. Enter on lower timeframe confirmation
  5. Consider price below both MAs for additional confirmation

  Exit Rules:
  1. Exit when MAs cross back (opposite signal)
  2. Use slower MA (200) as trailing stop
  3. Target minimum 2:1 or 3:1 risk-reward
  4. Exit if price breaks key support/resistance

  Risk Management:
  - Stop loss below swing low (longs) / above swing high (shorts)
  - Or stop beyond slower moving average
  - Risk 1-2% of account per trade

  Strengths:
  - Simple and clear signals
  - Works well in trending markets
  - Widely followed by institutions

  Weaknesses:
  - Lagging indicator (signals occur after trend starts)
  - False signals in ranging markets
  - Whipsaws in volatile sideways conditions

  Alternative Configurations:
  ┌────────────────┬────────────┬─────────────┬───────────┐
  │ Strategy Type  │  Short MA  │   Long MA   │ Timeframe │
  ├────────────────┼────────────┼─────────────┼───────────┤
  │ Classic        │ 50 SMA/EMA │ 200 SMA/EMA │ Daily     │
  ├────────────────┼────────────┼─────────────┼───────────┤
  │ Faster Signals │ 20 EMA     │ 50 SMA      │ Any       │
  ├────────────────┼────────────┼─────────────┼───────────┤
  │ Swing Trading  │ 20 EMA     │ 200 EMA     │ 4H/Daily  │
  └────────────────┴────────────┴─────────────┴───────────┘
  Sources:
  - https://www.investopedia.com/terms/g/goldencross.asp
  - https://acy.com/en/market-news/education/market-education-moving-average-golden-death-cross-strategy-j-o-20250723-125642/
  - https://www.vtmarkets.com/discover/golden-cross-trading-strategy/

  ---
  4. Bollinger Bands

  Standard Settings:
  - Middle Band: 20-period SMA
  - Upper Band: 20 SMA + (2 × 20-period standard deviation)
  - Lower Band: 20 SMA - (2 × 20-period standard deviation)

  Bollinger Band Squeeze Strategy:

  Squeeze Identification:
  - Bandwidth at multi-period low (volatility contraction)
  - Bands become parallel and flat
  - Bands may come inside Keltner Channels (strong squeeze signal)

  Entry Conditions:
  1. Identify squeeze phase (bands contracting)
  2. Enter on breakout: Price closes OUTSIDE the bands
  3. Look for momentum in breakout direction
  4. Confirm with higher-than-average volume

  Exit Conditions:
  1. Use trailing stops to lock in profits
  2. Exit when volatility expands rapidly after breakout
  3. Exit on reversal signals
  4. Take profit at predetermined risk-reward levels

  Overbought/Oversold Strategy:
  - Price touching upper band = Overbought (potential short)
  - Price touching lower band = Oversold (potential long)
  - Warning: Strong trends can ride bands for extended periods

  Sources:
  - https://www.investopedia.com/articles/technical/04/030304.asp
  - https://pyquantlab.medium.com/bollinger-band-squeeze-breakout-trading-strategy-with-trailing-stops-7aedc2f10958
  - https://quantra.quantinsti.com/glossary/How-to-Trade-a-Breakout-Strategy-using-Bollinger-Bands

  ---
  5. Stochastic Oscillator

  Standard Settings:
  - %K (Fast): 14-period
  - %D (Slow): 3-period SMA of %K

  Key Levels:
  ┌──────────┬────────────┬─────────────────────────┐
  │  Level   │ Condition  │         Signal          │
  ├──────────┼────────────┼─────────────────────────┤
  │ Above 80 │ Overbought │ Potential reversal down │
  ├──────────┼────────────┼─────────────────────────┤
  │ Below 20 │ Oversold   │ Potential reversal up   │
  └──────────┴────────────┴─────────────────────────┘
  Crossover Signals:
  - Bullish Crossover: %K crosses above %D (especially in oversold zone below 20)
  - Bearish Crossover: %K crosses below %D (especially in overbought zone above 80)

  Most Reliable Crossovers:
  - Occur within overbought (above 80) or oversold (below 20) zones
  - Combined with divergence patterns

  Divergence Strategy:
  - Price moves in one direction, oscillator moves in opposite
  - Signals potential trend reversal

  Sources:
  - https://www.oanda.com/us-en/trade-tap-blog/trading-knowledge/mastering-stochastic-oscillator-trading-strategies/
  - https://www.investopedia.com/terms/s/stochasticoscillator.asp
  - https://www.fpmarkets.com/en-ci/education/trading-guides/beginners-guide-to-stochastic-oscillator-trading-strategies/

  ---
  PART 2: ADVANCED TRADING SYSTEMS

  6. Donchian Channels / Turtle Trading System

  Donchian Channel Settings:
  - Fast Channel: 20-day high/low
  - Slow Channel: 55-day high/low

  Turtle Trading Entry Rules (LONG):
  1. Price closes above 20-day high (breakout)
  2. Enter LONG
  3. Pyramiding Rule: Add 50% of initial position size for each 0.5N favorable move

  Turtle Trading Entry Rules (SHORT):
  1. Price closes below 20-day low (breakdown)
  2. Enter SHORT
  3. Pyramiding Rule: Add 50% of initial position size for each 0.5N favorable move

  Stop Loss Rule:
  - Set at 2N from entry (N = ATR(20) × 1 unit)
  - Trail stop by 2N as position becomes profitable

  Exit Rules:
  1. Long Exit: Price closes below 10-day low
  2. Short Exit: Price closes above 10-day high

  Position Sizing:
  - 1 unit = (1% of account) / N
  - Maximum 4 units per position (pyramiding)

  System 1 (Shorter-term):
  - Entry: 20-day breakout
  - Exit: 10-day breakout against position

  System 2 (Longer-term):
  - Entry: 55-day breakout
  - Exit: 20-day breakout against position

  Sources:
  - https://medium.com/@FMZQuant/turtle-trading-strategy-based-on-donchian-channels-e235ef1bf0ea
  - https://www.altrady.com/blog/crypto-trading-strategies/turtle-trading-strategy-rules
  - https://trendspider.com/learning-center/donchian-channel-trading-strategies/

  ---
  7. Ichimoku Cloud Trading System

  Components:
  - Tenkan-sen (Conversion Line): (Highest High + Lowest Low) / 2 over 9 periods
  - Kijun-sen (Base Line): (Highest High + Lowest Low) / 2 over 26 periods
  - Senkou Span A: (Tenkan-sen + Kijun-sen) / 2, projected 26 periods ahead
  - Senkou Span B: (Highest High + Lowest Low) / 2 over 52 periods, projected 26 periods ahead
  - Kumo (Cloud): Space between Span A and Span B
  - Chikou Span (Lagging): Current close, plotted 26 periods behind

  Bullish Entry Conditions:
  1. Tenkan-sen crosses ABOVE Kijun-sen
  2. Price is ABOVE the Cloud
  3. Both Tenkan-sen and Kijun-sen above Kumo lines
  4. Chikou Span confirms (above price from 26 periods ago)
  5. Upper Kumo line above lower Kumo line (bullish cloud)

  Bearish Entry Conditions:
  1. Tenkan-sen crosses BELOW Kijun-sen
  2. Price is BELOW the Cloud
  3. Both Tenkan-sen and Kijun-sen below Kumo lines
  4. Chikou Span confirms (below price from 26 periods ago)
  5. Lower Kumo line above upper Kumo line (bearish cloud)

  Exit Signals:
  - Opposite Tenkan-sen/Kijun-sen crossover
  - Price crossing back through Cloud in opposite direction
  - Chikou Span losing confirmation

  Best Practices:
  - Use multiple confirmations, not just crossover
  - Cloud thickness indicates support/resistance strength
  - Trade in direction of overall trend

  Sources:
  - https://trendspider.com/learning-center/ichimoku-cloud-trading-strategies/
  - https://www.oanda.com/us-en/trade-tap-blog/analysis/technical/ichimoku-cloud-trading-guide-key-strategies/

  ---
  8. Parabolic SAR (Stop and Reverse)

  Purpose: Trend-following indicator providing entry, exit, and trailing stop signals

  Visual Signals:
  - Dots below price: Uptrend (bullish)
  - Dots above price: Downtrend (bearish)

  Entry Rules:
  - Long Entry: When SAR dots move below price
  - Short Entry: When SAR dots move above price

  Exit Rules:
  - Long Position Exit: When price crosses below SAR dots
  - Short Position Exit: When price crosses above SAR dots

  Stop and Reverse Feature:
  - When stopped out, signal suggests entering opposite direction

  Best Used In:
  - Strongly trending markets
  - With additional trend confirmation

  Avoid Using In:
  - Ranging/sideways markets (generates whipsaws)

  Sources:
  - https://www.gomarkets.com/en-au/articles/strategy-series-mastering-the-parabolic-sar-in-trading-entry-and-exit
  - https://www.cmcmarkets.com/en-gb/technical-analysis/parabolic-sar

  ---
  9. Heikin-Ashi Candlestick Strategy

  Heikin-Ashi Formula:
  - HA Close: (Open + High + Low + Close) / 4
  - HA Open: (Previous HA Open + Previous HA Close) / 2
  - HA High: Max(High, HA Open, HA Close)
  - HA Low: Min(Low, HA Open, HA Close)

  Trend Detection:
  - Uptrend: Series of green HA candles with no lower wicks
  - Downtrend: Series of red HA candles with no upper wicks
  - Consolidation: Small candles with wicks on both sides

  Entry Rules:
  - Long Entry: When HA candles turn green after red series
  - Short Entry: When HA candles turn red after green series

  Exit Rules:
  - Long Exit: When HA candle shows lower wick or turns red
  - Short Exit: When HA candle shows upper wick or turns green

  Best Practice:
  - Use HA for trend direction, regular candles for entry timing

  Sources:
  - https://www.investopedia.com/trading/heikin-ashi-better-candlestick/
  - https://www.luxalgo.com/blog/heikin-ashi-candles-smooth-trend-detection/
  - https://www.binance.com/en/square/post/17811524895185

  ---
  PART 3: VOLUME-BASED INDICATORS

  10. ATR (Average True Range) - Volatility & Position Sizing

  Standard Setting: 14-period

  Primary Uses:
  1. Position sizing
  2. Stop loss placement
  3. Identifying volatility regimes

  Position Sizing Formula:
  Position Size = Account Risk / (ATR × Multiplier)

  Stop Loss Placement:
  - Conservative: 1 × ATR from entry
  - Moderate: 2 × ATR from entry
  - Wide: 3 × ATR from entry

  ATR Channel Breakout Strategy:
  - Create channels around price using ATR
  - Enter when price penetrates ATR channels
  - Trade both long and short breakouts

  Risk Management:
  - Lower position sizes during high volatility (high ATR)
  - Increase position sizes during low volatility (low ATR)

  Sources:
  - https://www.ig.com/en/trading-strategies/what-is-the-average-true-range--atr--indicator-and-how-do-we-tr-240905
  - https://www.vtmarkets.com/discover/average-true-range-atr-indicator-guide-master-volatility-trading/
  - https://www.luxalgo.com/blog/5-position-sizing-methods-for-high-volatility-trades/

  ---
  11. Money Flow Index (MFI)

  Standard Setting: 14-period

  Key Levels:
  ┌──────────┬────────────┬─────────────────────────┐
  │  Level   │ Condition  │         Signal          │
  ├──────────┼────────────┼─────────────────────────┤
  │ Above 80 │ Overbought │ Potential reversal down │
  ├──────────┼────────────┼─────────────────────────┤
  │ Below 20 │ Oversold   │ Potential reversal up   │
  └──────────┴────────────┴─────────────────────────┘
  Difference from RSI:
  - MFI incorporates BOTH price AND volume
  - RSI only uses price

  Entry Rules:
  1. MFI drops below 20 → Look for long entry on cross back above 20
  2. MFI rises above 80 → Look for short entry on cross back below 80

  Divergence Strategy:
  - Bullish Divergence: Price lower low, MFI higher low
  - Bearish Divergence: Price higher high, MFI lower high

  Best For:
  - Identifying trend pullback completion in trending markets
  - Confirming reversal signals with volume confirmation

  Sources:
  - https://www.investopedia.com/terms/m/mfi.asp
  - https://www.ig.com/en/trading-strategies/what-is-the-money-flow-index-and-how-does-it-work--190618
  - https://www.tradingsim.com/blog/money-flow-index

  ---
  12. VWAP (Volume Weighted Average Price)

  Calculation:
  VWAP = Cumulative(Price × Volume) / Cumulative Volume

  Institutional Benchmark:
  - Used by institutions to assess execution quality
  - Influences intraday price action

  Trading Strategy:

  Long Entry:
  - Price pulls back to VWAP from above
  - Shows rejection/bounce at VWAP
  - Enter on confirmation candle

  Short Entry:
  - Price rallies to VWAP from below
  - Shows rejection at VWAP
  - Enter on confirmation candle

  Best Used For:
  - Day trading (intraday)
  - Identifying fair value
  - Support/resistance levels

  Combine With:
  - Volume profile for key levels
  - Other indicators for confirmation

  Sources:
  - https://www.investopedia.com/ask/answers/031115/what-common-strategy-traders-implement-when-using-volume-weighted-average-price-vwap.asp
  - https://www.wrightresearch.in/blog/understanding-volume-weighted-average-price-vwap-trading-strategy/
  - https://www.luxalgo.com/blog/volume-based-support-and-resistance-explained/

  ---
  PART 4: MOMENTUM INDICATORS

  13. Williams %R

  Standard Setting: 14-period

  Range: 0 to -100

  Key Levels:
  ┌─────────────┬────────────┬─────────────────────────┐
  │    Level    │ Condition  │         Signal          │
  ├─────────────┼────────────┼─────────────────────────┤
  │ 0 to -20    │ Overbought │ Potential reversal down │
  ├─────────────┼────────────┼─────────────────────────┤
  │ -80 to -100 │ Oversold   │ Potential reversal up   │
  └─────────────┴────────────┴─────────────────────────┘
  Entry Rules:
  - Buy: When %R moves FROM oversold (-80 to -100) TO above -80
  - Sell: When %R moves FROM overbought (0 to -20) TO below -20

  Key Point:
  Wait for indicator to EXIT extreme zones, not just enter them

  Works Best In:
  - Sideways/ranging markets
  - Oscillating price action between support and resistance

  Sources:
  - https://www.luxalgo.com/blog/williams-r-indicator-oscillator-for-overbought-oversold-levels/
  - https://commodity.com/technical-analysis/williams-r/
  - https://www.fidelity.com/learning-center/trading-investing/technical-analysis/technical-indicator-guide/williams-r

  ---
  14. ADX (Average Directional Index)

  Components:
  - ADX Line: Measures trend strength (not direction)
  - +DI: Measures upward movement
  - -DI: Measures downward movement

  Trend Strength Levels:
  ┌───────────┬───────────────────┬────────────────────────┐
  │ ADX Level │  Interpretation   │         Action         │
  ├───────────┼───────────────────┼────────────────────────┤
  │ Below 20  │ Weak/No trend     │ Avoid trading          │
  ├───────────┼───────────────────┼────────────────────────┤
  │ 20-25     │ Emerging trend    │ Watch for confirmation │
  ├───────────┼───────────────────┼────────────────────────┤
  │ 25-40     │ Strong trend      │ Trade in direction     │
  ├───────────┼───────────────────┼────────────────────────┤
  │ 40-60     │ Very strong trend │ Strong trades          │
  ├───────────┼───────────────────┼────────────────────────┤
  │ 60+       │ Extreme strength  │ Potential blow-off     │
  └───────────┴───────────────────┴────────────────────────┘
  Entry Rules:
  - Long Entry: +DI crosses above -DI AND ADX is rising (above 25)
  - Short Entry: -DI crosses above +DI AND ADX is rising (above 25)

  Exit Rules:
  - Falling ADX indicates trend weakening
  - ADX dropping below 25 = exit or reduce position

  Popular Combinations:
  1. ADX + RSI: ADX confirms trend, RSI shows overbought/oversold
  2. ADX + Moving Averages: Trend strength + price location
  3. ADX + MACD: Trend strength + momentum

  Key Benefit:
  Avoids trading in weak/ranging markets

  Sources:
  - https://www.investopedia.com/articles/trading/07/adx-trend-indicator.asp
  - https://tradingstrategyguides.com/adx-indicator/
  - https://www.schwab.com/learn/story/spot-and-stick-to-trends-with-adx-and-rsi

  ---
  15. Elder's Force Index

  Developer: Dr. Alexander Elder

  Formula:
  Force Index = (Close - Previous Close) × Volume

  Standard Settings:
  - EFI(1): 1-period EMA (highly sensitive)
  - EFI(13): 13-period EMA (intermediate)
  - EFI(2): 2-period EMA (short-term)

  Interpretation:
  ┌───────────────────────┬────────────────────────────┐
  │ Force Index Direction │     Market Implication     │
  ├───────────────────────┼────────────────────────────┤
  │ Positive & Rising     │ Strong bullish momentum    │
  ├───────────────────────┼────────────────────────────┤
  │ Positive & Falling    │ Bullish momentum weakening │
  ├───────────────────────┼────────────────────────────┤
  │ Negative & Falling    │ Strong bearish momentum    │
  ├───────────────────────┼────────────────────────────┤
  │ Negative & Rising     │ Bearish momentum weakening │
  └───────────────────────┴────────────────────────────┘
  Entry Rules (Long):
  1. Force Index turns negative (minor dip)
  2. Force Index crosses back to positive
  3. Price trend is up
  4. Enter on confirmation

  Entry Rules (Short):
  1. Force Index turns positive (minor rally)
  2. Force Index crosses back to negative
  3. Price trend is down
  4. Enter on confirmation

  Divergence Strategy:
  - Price makes new high, Force Index makes lower high → Bearish
  - Price makes new low, Force Index makes higher low → Bullish

  Best For:
  - Confirming trend strength
  - Identifying potential reversals
  - Measuring buying/selling pressure

  Sources:
  - https://chartschool.stockcharts.com/table-of-contents/technical-indicators-and-overlays/technical-indicators/force-index
  - https://www.investopedia.com/articles/trading/03/031203.asp
  - https://www.luxalgo.com/blog/elders-force-index-indicator-quantifying-market-force-and-direction/

  ---
  PART 5: STATISTICAL ARBITRAGE SYSTEMS

  16. Pair Trading (Cointegration-Based)

  Concept: Market-neutral strategy matching long position in one asset with short position in another

  Step 1: Pair Identification
  - Use Engle-Granger test or Johansen test for cointegration
  - Select pairs with p-value < 0.05 (statistically significant)
  - Calculate correlation and Hurst exponent

  Step 2: Spread Calculation
  Spread = Price_A - (Hedge_Ratio × Price_B)
  - Hedge ratio from OLS regression or cointegration coefficient

  Step 3: Z-Score Normalization
  Z-Score = (Spread - Mean) / Standard_Deviation
  - Use rolling window (20-60 days)

  Entry Rules (Standard):
  ┌─────────────┬───────────────────┬──────────────────────────────────────────────────┐
  │   Signal    │ Z-Score Condition │                      Action                      │
  ├─────────────┼───────────────────┼──────────────────────────────────────────────────┤
  │ Long Entry  │ Z ≤ -2.0          │ Long spread (buy undervalued, short overvalued)  │
  ├─────────────┼───────────────────┼──────────────────────────────────────────────────┤
  │ Short Entry │ Z ≥ +2.0          │ Short spread (short overvalued, buy undervalued) │
  └─────────────┴───────────────────┴──────────────────────────────────────────────────┘
  Exit Rules:
  - Standard Exit: Z-score crosses 0
  - Conservative Exit: Z-score reaches ±1.0
  - Stop Loss: Z-score extends beyond ±3-4

  Threshold Variations:
  - Conservative: Entry at ±2.5, Exit at ±0.5
  - Aggressive: Entry at ±1.5, Exit at ±0

  Position Sizing:
  - Adjust based on volatility
  - Consider half-life of mean reversion

  Sources:
  - https://blog.quantinsti.com/epat-project-mean-reversion-statistical-arbitrage-pair-trading-strategy-indian-market-sectors/
  - https://arnav04g.medium.com/trading-of-cointegration-pairs-using-mean-reversion-statistical-arbitrage-cbc13b70bc3d
  - https://databento.com/docs/examples/algo-trading/pairs-trading

  ---
  17. Dual Momentum Strategy

  Developer: Gary Antonacci

  Concept: Combines relative and absolute momentum using 12-month lookback

  The Two-Step Process:

  Step 1: Absolute Momentum (Trend Filter)
  - Calculate 12-month return for broad market (S&P 500)
  - If return POSITIVE → Proceed to Step 2
  - If return NEGATIVE → Move to defensive assets (bonds/cash)

  Step 2: Relative Momentum (Asset Selection)
  - Compare 12-month returns of candidate assets
  - Select asset with highest return
  - Invest in top performer

  Example Implementation:
  1. Compare U.S. stocks vs International stocks (12-month returns)
  2. Only invest if S&P 500 shows positive absolute momentum
  3. Choose asset class with superior relative strength
  4. Move to bonds/cash when absolute momentum turns negative

  Rebalancing: Typically monthly

  Advantages:
  - Significantly increases risk-adjusted returns
  - Provides downside protection
  - Simple 3-ETF implementation

  Sources:
  - https://www.optimalmomentum.com/dual-relative-absolute-momentum/
  - https://seekingalpha.com/article/2649915-dual-momentum-how-to-implement-strategy-for-higher-returns-with-lower-risk
  - https://www.composer.trade/learn/dual-momentum-explained

  ---
  PART 6: PRICE ACTION STRATEGIES

  18. Break & Retest Strategy

  Core Concept: Trade breakouts of key levels after price retests them

  Setup:
  1. Identify clear support or resistance level
  2. Wait for clean breakout (strong candle close beyond level)
  3. Wait for retest back to the level
  4. Enter on rejection from retest

  Entry Rules:
  1. Breakout: Strong candle close beyond key level
  2. Confirmation: Increased volume during breakout
  3. Retest: Price returns to broken level
  4. Entry: Look for rejection patterns (pin bars, engulfing candles)

  Retest Validation:
  - Support becomes resistance (or vice versa)
  - Rejection at retest confirms validity
  - Price holds the level

  Exit Rules:
  - Next major support/resistance level
  - 1:2 or 1:3 risk-reward minimum
  - Opposite structure break

  Confirmation Rules:
  1. Strong candle close beyond key level
  2. Increased volume during breakout
  3. Follow-up candle holds the level
  4. Price action confirms

  Sources:
  - https://www.chartfanatics.com/strategies/break-retest-strategy
  - https://capital.com/en-int/analysis/how-to-trade-the-break-retest
  - https://www.binance.com/en/square/post/28037941565722

  ---
  19. Multiple Timeframe Analysis (Daily-4H-1H)

  Timeframe Structure:
  ┌───────────┬─────────┬───────────────────────────────────┐
  │ Timeframe │  Role   │              Purpose              │
  ├───────────┼─────────┼───────────────────────────────────┤
  │ Daily     │ Terrain │ Overall trend, key levels         │
  ├───────────┼─────────┼───────────────────────────────────┤
  │ 4H        │ Path    │ Intermediate structure, pullbacks │
  ├───────────┼─────────┼───────────────────────────────────┤
  │ 1H        │ Trigger │ Precise entry/exit points         │
  └───────────┴─────────┴───────────────────────────────────┘
  Analysis Workflow:

  Step 1: Daily Chart
  - Identify overall trend direction
  - Mark key support and resistance levels
  - Identify major swing highs and lows
  - Determine market structure (bullish/bearish/ranging)

  Step 2: 4H Chart
  - Look for pullbacks in Daily trend direction
  - Identify smaller support/resistance zones
  - Confirm trend continuation patterns
  - Monitor for price action signals

  Step 3: 1H Chart
  - Find precise entry triggers when 4H and Daily align
  - Look for confirmation signals (candlestick patterns, price action)
  - Set optimal stop loss and take profit levels
  - Execute trades with favorable risk-reward ratios

  Confluence Entry Rules:
  1. Trend alignment: 1H trade in direction of 4H and Daily trends
  2. Level confluence: Key levels from Daily AND 4H charts align
  3. Multiple indicator alignment: Fibonacci, MAs, S/R from different timeframes coincide
  4. Price action confirmation: Rejection candles, pin bars, engulfing patterns on 1H

  Exit Rules:
  - Take Profit: Next Daily S/R level or 4H key structural level
  - Risk-Reward: Minimum 1:2 or 1:3
  - Stop Loss: Below/above 1H swing structure, beyond 4H key levels

  Sources:
  - https://tradeciety.com/how-to-perform-a-multiple-time-frame-analysis
  - https://www.learntotradethemarket.com/forex-trading-strategies/how-to-use-1-4-hour-charts-to-confirm-daily-chart-signals
  - https://www.mindmathmoney.com/articles/multi-timeframe-analysis-trading-strategy-the-complete-guide-to-trading-multiple-timeframes

  ---
  PART 7: PERFORMANCE METRICS & VALIDATION

  Key Backtesting Metrics

  Essential Metrics:
  1. CAGR (Compound Annual Growth Rate): Annualized return
  2. Maximum Drawdown (MDD): Largest peak-to-trough decline
  3. Sharpe Ratio: Risk-adjusted returns (Return - Risk-Free) / Volatility
  4. Sortino Ratio: Similar to Sharpe but only considers downside volatility
  5. Win Rate: Percentage of profitable trades
  6. Profit Factor: Gross profits / Gross losses
  7. Average Win/Loss Ratio: Average winner size vs average loser size

  Performance Benchmarks (from real backtests):
  - 30-Year Strategy: 10.74% CAGR, 0.86 Sharpe, 25.13% Max DD
  - 15-Year SPY Strategy: 40% CAGR, 23.5% Max DD

  Walk Forward Analysis:
  - Determines optimal parameters
  - Assesses strategy robustness
  - Uses out-of-sample data
  - Simulates real-world trading

  Best Practices:
  - Compare CAGR to mean drawdown, not just max drawdown
  - Use Monte Carlo simulation for drawdown variations
  - Account for transaction costs and slippage
  - Consider psychological tolerance for drawdowns

  Sources:
  - https://www.interactivebrokers.com/campus/ibkr-quant-news/the-future-of-backtesting-a-deep-dive-into-walk-forward-analysis/
  - https://www.luxalgo.com/blog/top-7-metrics-for-backtesting-results/
  - https://medium.com/@yavuzakbay/how-to-evaluate-a-trading-strategy-like-a-quant-fc903e093015

  ---
  PART 8: PROVEN STRATEGY COMBINATIONS

  High-Probability Indicator Combinations

  1. RSI + MACD + Volume (65-75% accuracy cited)
  - Use MACD for trend direction
  - Use RSI for entry/exit timing
  - Use Volume for confirmation

  2. Moving Average + Stochastic
  - MA for trend direction
  - Stochastic for overbought/oversold entries

  3. Bollinger Bands + Stochastic
  - BB for volatility and squeeze
  - Stochastic for timing entries in squeeze

  4. SMA + OBV + CCI
  - SMA for trend
  - OBV for volume confirmation
  - CCI for momentum

  5. ADX + RSI Combination
  - ADX confirms trend presence/strength
  - RSI identifies pullback entry points
  - Only enter when ADX > 25 (strong trend)

  6. Price Action + Market Structure + Liquidity
  - No indicators needed
  - Focus on support/resistance levels
  - Trade breakouts and retests
  - Identified as most effective by 3-year trader

  Sources:
  - https://www.reddit.com/r/Daytrading/comments/1jmzyt3/the_best_trading_strategy_after_3_years_heres/
  - https://www.ubishops.ca/wp-content/uploads/he20240105.pdf
  - https://extractalpha.com/2025/07/01/top-7-trading-signals-every-quant-should-track/

  ---
  PART 9: TRADING STRATEGY TYPES

  Mean Reversion vs Momentum vs Trend Following
  ┌─────────────────┬─────────────────────────┬───────────────────────────────────────────────┬───────────────────────────────────┐
  │  Strategy Type  │  Best Market Condition  │                Key Indicators                 │             Approach              │
  ├─────────────────┼─────────────────────────┼───────────────────────────────────────────────┼───────────────────────────────────┤
  │ Mean Reversion  │ Range-bound markets     │ RSI, Bollinger Bands, Stochastic              │ Buy low, sell high at extremes    │
  ├─────────────────┼─────────────────────────┼───────────────────────────────────────────────┼───────────────────────────────────┤
  │ Momentum        │ Strong trending markets │ MACD, ADX, Moving Averages                    │ Ride trends, let profits run      │
  ├─────────────────┼─────────────────────────┼───────────────────────────────────────────────┼───────────────────────────────────┤
  │ Trend Following │ Established trends      │ Moving Averages, Donchian Channels, Breakouts │ Follow trends, use trailing stops │
  └─────────────────┴─────────────────────────┴───────────────────────────────────────────────┴───────────────────────────────────┘
  Market Condition Selection:
  - Mean-reversion excels in range-bound markets
  - Momentum thrives in trending markets
  - Strategy selection depends on market conditions and risk tolerance
  - Combining strategies provides diversification benefits

  Sources:
  - https://bookmap.com/blog/key-algorithmic-trading-strategies-from-trend-following-to-mean-reversion-and-beyond
  - https://www.quantinsti.com/articles/types-trading-strategies/
  - https://www.interactivebrokers.com/campus/ibkr-quant-news/mean-reversion-strategies-introduction-trading-strategies-and-more-part-i/

  ---
  PART 10: BEST PRACTICES & WARNINGS

  Universal Principles

  1. Risk Management First
  - Risk 1-2% of account per trade
  - Always use stop losses
  - Calculate position size based on volatility

  2. Confirmation is Key
  - Never rely on single indicator
  - Use multiple timeframe analysis
  - Wait for price action confirmation

  3. Avoid Common Mistakes
  - RSI can remain overbought/oversold in strong trends
  - Moving averages lag (signals occur after trend starts)
  - False breakouts are common
  - Whipsaws occur in ranging markets

  4. Strategy Development
  - Backtest thoroughly
  - Use walk-forward analysis
  - Account for transaction costs
  - Paper trade first

  5. Market Adaptation
  - No single strategy works in all conditions
  - Adjust parameters based on volatility
  - Switch strategies based on market regime
  - Combine complementary approaches

  Proven 3+ Year Systems Summary

  Most Reliable Long-Term Systems:
  1. Golden/Death Cross (50/200 MA) - Decades of use, institutionally followed
  2. Turtle Trading (Donchian 20/55) - 40+ years of use, originally taught to beginners
  3. Dual Momentum (12-month) - Proven by Gary Antonacci, extensive backtesting
  4. Pair Trading (Cointegration) - Statistical arbitrage, market-neutral
  5. Break & Retest (Price Action) - Timeless, based on market psychology

  Highest Accuracy Combinations (cited):
  1. RSI + MACD + Volume (65-75%)
  2. ADX + RSI (trend filter + timing)
  3. Multiple Timeframe Analysis (D-4H-1H)
  4. Price Action + Structure + Liquidity (no indicators needed)

  ---
  CONCLUSION

  This comprehensive research has identified the most proven trading systems used by traders for 3+ years. The key findings are:

  1. Simplicity wins - Price action, market structure, and liquidity often outperform complex indicator combinations
  2. Combination is superior - Multiple indicators working together provide better signals than any single indicator
  3. Context matters - The same strategy performs differently in trending vs ranging markets
  4. Risk management is essential - All profitable systems have strict position sizing and stop loss rules
  5. Backtesting validation - Walk-forward analysis and out-of-sample testing are crucial for strategy validation

  For your codebase (strategy.go with RSI at line 139): The research confirms RSI is one of the most reliable indicators, especially when:
  - Combined with trend filters (ADX, Moving Averages)
  - Used with divergence patterns
  - Applied in multiple timeframe analysis
  - Confirmed with volume and price action

  ---
  PART 11: PROVEN STRATEGY PERFORMANCE DATA

  Strategy Performance Summary Table
  ┌──────────────────────────────────┬─────────┬───────┬──────────┬─────────────────┬─────────────────────────────────────┐
  │ Strategy                         │ Started │ CAGR  │ Max DD   │ Monthly Return │ Notes                                  │
  ├──────────────────────────────────┼─────────┼───────┼──────────┼─────────────────┼─────────────────────────────────────┤
  │ Golden Cross (50/200 MA)         │ 1950s   │ 5-7%  │ ~30%     │ ~0.5%          │ Lagging but reliable                   │
  ├──────────────────────────────────┼─────────┼───────┼──────────┼─────────────────┼─────────────────────────────────────┤
  │ Turtle Trading (Original)        │ 1983    │ 80%   │ N/A      │ ~6.7%          │ Original 4 years only                  │
  ├──────────────────────────────────┼─────────┼───────┼──────────┼─────────────────┼─────────────────────────────────────┤
  │ Turtle Trading (Modern)          │ 2005+   │ 2-32% │ 17-42%   │ ~0.2-2.7%      │ Significantly degraded                  │
  ├──────────────────────────────────┼─────────┼───────┼──────────┼─────────────────┼─────────────────────────────────────┤
  │ Dual Momentum GEM                │ 1973    │ 12-17%│ ~25%     │ ~1.0-1.4%      │ Consistent outperformance              │
  ├──────────────────────────────────┼─────────┼───────┼──────────┼─────────────────┼─────────────────────────────────────┤
  │ RSI 30-70 (5-day)                │ 1985    │ Poor  │ High     │ Negative        │ Not profitable standalone              │
  ├──────────────────────────────────┼─────────┼───────┼──────────┼─────────────────┼─────────────────────────────────────┤
  │ RSI Modified (Cumulative)        │ 1999    │ 30.3% │ 35%      │ ~2.5%          │ Enhanced version                       │
  ├──────────────────────────────────┼─────────┼───────┼──────────┼─────────────────┼─────────────────────────────────────┤
  │ Bollinger Band Squeeze           │ 1980s   │ Var.  │ Variable │ Variable       │ Market-dependent                       │
  ├──────────────────────────────────┼─────────┼───────┼──────────┼─────────────────┼─────────────────────────────────────┤
  │ 30-Year Backtested System        │ ~1990   │ 10.74%│ 25.13%   │ ~0.9%          │ Reddit algotrading                     │
  ├──────────────────────────────────┼─────────┼───────┼──────────┼─────────────────┼─────────────────────────────────────┤
  │ 15-Year SPY Strategy             │ ~2005   │ 40%   │ 23.5%    │ ~3.3%          │ TradingWhale                           │
  └──────────────────────────────────┴─────────┴───────┴──────────┴─────────────────┴─────────────────────────────────────┘

  Detailed Monthly Returns by Strategy
  ┌──────────────────────────────┬─────────────────┬─────────────────────────┐
  │ Strategy                     │ Monthly Return  │ Calculation             │
  ├──────────────────────────────┼─────────────────┼─────────────────────────┤
  │ Golden Cross                 │ ~0.42%/month    │ 5% CAGR ÷ 12           │
  ├──────────────────────────────┼─────────────────┼─────────────────────────┤
  │ Turtle Trading (Original)    │ ~6.7%/month     │ 80% CAGR ÷ 12 (original)│
  ├──────────────────────────────┼─────────────────┼─────────────────────────┤
  │ Turtle Trading (Modern)      │ 0.17-2.7%/month │ 2-32% CAGR ÷ 12         │
  ├──────────────────────────────┼─────────────────┼─────────────────────────┤
  │ Dual Momentum GEM            │ 1.0-1.4%/month  │ 12-17% CAGR ÷ 12        │
  ├──────────────────────────────┼─────────────────┼─────────────────────────┤
  │ RSI 30-70 Basic              │ Negative        │ Not profitable          │
  ├──────────────────────────────┼─────────────────┼─────────────────────────┤
  │ RSI Cumulative               │ ~2.5%/month     │ 30.3% CAGR ÷ 12         │
  ├──────────────────────────────┼─────────────────┼─────────────────────────┤
  │ 30-Year System               │ ~0.9%/month     │ 10.74% CAGR ÷ 12        │
  ├──────────────────────────────┼─────────────────┼─────────────────────────┤
  │ 15-Year SPY System           │ ~3.3%/month     │ 40% CAGR ÷ 12           │
  └──────────────────────────────┴─────────────────┴─────────────────────────┘

  CAGR Definition and Calculation

  CAGR = Compound Annual Growth Rate
  Measures the mean annual growth rate of an investment over a specified period (>1 year)

  Formula:
  CAGR = (Ending Value / Beginning Value)^(1/n) - 1
  Where n = Number of years

  Example:
  Invest $10,000 → Grows to $16,105 after 5 years
  CAGR = ($16,105 / $10,000)^(1/5) - 1 = 10% per year

  Monthly Return from CAGR:
  Monthly = (1 + CAGR)^(1/12) - 1

  ┌───────┬─────────────────┐
  │ CAGR  │ Monthly Return  │
  ├───────┼─────────────────┤
  │ 10%   │ ~0.8%/month     │
  ├───────┼─────────────────┤
  │ 12%   │ ~0.95%/month    │
  ├───────┼─────────────────┤
  │ 20%   │ ~1.53%/month    │
  ├───────┼─────────────────┤
  │ 40%   │ ~2.84%/month    │
  └───────┴─────────────────┘

  Key Performance Metrics Explained:
  - CAGR: Annualized return (compounded)
  - Max Drawdown: Largest peak-to-trough decline (risk)
  - Sharpe Ratio: Return per unit of risk (Risk-adjusted performance)
  - Win Rate: Percentage of profitable trades
  - Profit Factor: Gross profits / Gross losses

  Sources:
  - https://www.investopedia.com/terms/c/cagr.asp
  - https://www.quantifiedstrategies.com/golden-cross-trading-strategy/
  - https://www.gate.com/learn/articles/turtle-trading-rules-classic-system-with-annual-returns-up-to-62-71/10867
  - https://www.optimalmomentum.com/extended-backtest-of-global-equities-momentum/
  - https://www.quantifiedstrategies.com/70-30-rsi-trading-strategy/
  - https://www.reddit.com/r/algotrading/comments/1neqe71/30year_backtesting_1074_cagr_086_sharpe_2513_maxdd/

  ---
  PART 12: YOUR SYSTEM VS PROVEN STRATEGIES

  Your Current Trading System Analysis

  Indicators in Your System (60+ features):
  ┌─────────────────┬──────────────────────────────────────────────────────────────────────────┐
  │ Category        │ Indicators                                                                 │
  ├─────────────────┼──────────────────────────────────────────────────────────────────────────┤
  │ Trend Following │ EMA(10,20,50,100,200), HMA(9,20,50,100,200), Kijun26, ROC(5,10,20), SLOPE_20│
  ├─────────────────┼──────────────────────────────────────────────────────────────────────────┤
  │ Momentum        │ RSI(7,14,21), MACD(12,26,9), MFI(14), ROC5                              │
  ├─────────────────┼──────────────────────────────────────────────────────────────────────────┤
  │ Volatility      │ ATR(7,14), BB(20,50), VOLRET_20, HighLowDiff, RangeWidth                 │
  ├─────────────────┼──────────────────────────────────────────────────────────────────────────┤
  │ Volume          │ OBV, VolSMA(20,50), VolEMA(20,50), VolZ(20,50), VolPerTrade, BuyRatio   │
  ├─────────────────┼──────────────────────────────────────────────────────────────────────────┤
  │ Trend Strength  │ ADX(14), PlusDI, MinusDI                                                   │
  ├─────────────────┼──────────────────────────────────────────────────────────────────────────┤
  │ Market Structure│ SwingHigh/Low, BOS, FVG, Sweep, Displacement, HH/LL(20,50,100,200)        │
  ├─────────────────┼──────────────────────────────────────────────────────────────────────────┤
  │ Price Action    │ Body, BodyPct, WickUpPct, WickDownPct, ClosePos, COMPRESSION             │
  ├─────────────────┼──────────────────────────────────────────────────────────────────────────┤
  │ Regime Filter   │ ATR14_SMA50 (volatility filter)                                           │
  └─────────────────┴──────────────────────────────────────────────────────────────────────────┘

  Your System's Advantages:
  1. Massive Feature Set (60+ vs 3-5 in traditional strategies)
  2. Automated Evolution via genetic algorithms
  3. Robust Validation with walk-forward analysis
  4. Advanced Features: Market Structure Detection (BOS, FVG, Sweep)
  5. Volume Profile Analysis: BuyRatio, Imbalance, VolZ
  6. Candle Anatomy: BodyPct, WickUpPct, WickDownPct
  7. Volatility Regime Filter: Only trades when ATR14 > ATR14_SMA50

  Your System's Disadvantages:
  1. Curve-Fitting Risk: 60+ features increase overfitting potential
  2. Execution Complexity: Harder to implement in live trading
  3. Lack of Long-Term Track Record: No documented live trading results
  4. Computational Requirements: Significant resources needed

  Performance Targets to Beat Proven Strategies
  ┌─────────────────────┬──────────┬─────────────┬────────────────────────────────────────┐
  │ Metric              │ Target   │ Benchmark   │ Rationale                               │
  ├─────────────────────┼──────────┼─────────────┼────────────────────────────────────────┤
  │ CAGR                │ > 12%    │ Dual Mom.   │ Beat best proven strategy (12-17%)     │
  ├─────────────────────┼──────────┼─────────────┼────────────────────────────────────────┤
  │ Maximum Drawdown    │ < 30%    │ Dual Mom.   │ Manageable risk for most traders        │
  ├─────────────────────┼──────────┼─────────────┼────────────────────────────────────────┤
  │ Sharpe Ratio        │ > 1.0    │ Institutional │ Good risk-adjusted returns            │
  ├─────────────────────┼──────────┼─────────────┼────────────────────────────────────────┤
  │ Monthly Return      │ 1-2%     │ Consistent  │ Sustainable growth without excessive risk│
  ├─────────────────────┼──────────┼─────────────┼────────────────────────────────────────┤
  │ Win Rate            │ > 45%    │ Trend Foll. │ With 1.5+ reward:risk ratio             │
  └─────────────────────┴──────────┴─────────────┴────────────────────────────────────────┘

  Recommendations to Improve Your System:
  1. Feature Selection Reduction: Limit to 10-15 most predictive features
  2. Simplify Rules: Maximum 3-4 conditions per strategy
  3. Add Proven Strategy Baselines: Include Golden Cross, Dual Momentum for comparison
  4. Paper Trading: Test for 6-12 months before live deployment
  5. Target Performance: Beat 12% CAGR with under 30% max drawdown

  Best Proven Strategies to Emulate (Ranked):
  ┌──────┬────────────────────┬─────────────────────────────────────────────────────────────────┐
  │ Rank │ Strategy          │ Why Copy It                                                      │
  ├──────┼────────────────────┼─────────────────────────────────────────────────────────────────┤
  │ 1    │ Dual Momentum GEM │ 40+ year track record, 12-17% CAGR, simple rules                 │
  ├──────┼────────────────────┼─────────────────────────────────────────────────────────────────┤
  │ 2    │ Golden Cross      │ Decades of use, reliable signals, institutional following       │
  ├──────┼────────────────────┼─────────────────────────────────────────────────────────────────┤
  │ 3    │ Modified RSI      │ 30.3% CAGR, incorporates volume, adapts basic RSI               │
  ├──────┼────────────────────┼─────────────────────────────────────────────────────────────────┤
  │ 4    │ Bollinger Squeeze │ Captures volatility breakouts, works in all markets              │
  └──────┴────────────────────┴─────────────────────────────────────────────────────────────────┘

  ---
  PART 13: ADDITIONAL ADVANCED STRATEGIES & INDICATORS

  Keltner Channels & BB-Keltner Squeeze Strategy

  What is Keltner Channel:
  - Volatility-based indicator similar to Bollinger Bands
  - Uses ATR (Average True Range) instead of standard deviation
  - Middle line: EMA (typically 20-period)
  - Upper/Lower bands: EMA ± (ATR × multiplier, typically 2)

  BB-Keltner Squeeze Strategy:
  - Squeeze Identification: Bollinger Bands move INSIDE Keltner Channels
  - Both upper and lower BB must be inside KC for true squeeze
  - Signals low volatility period before explosive breakout

  Entry Rules:
  1. Wait for squeeze to form (BB inside KC)
  2. Enter on breakout when price closes outside either band
  3. Confirm with volume expansion
  4. Use ATR trailing stops for risk management

  Exit Rules:
  - Use ATR-based trailing stops (typically 2-3 × ATR)
  - Exit when volatility expands rapidly after breakout
  - Take profit at predetermined risk-reward levels (2:1 or 3:1)

  Sources:
  - https://www.chartguys.com/articles/keltner-channel
  - https://www.pyquantlab.com/article.php?file=Bollinger-Keltner%2520Squeeze%2520Breakout%2520Trading%2520Strategy
  - https://trendspider.com/learning-center/bb-kc-squeeze-a-powerful-indicator-for-trading-range-breakouts/
  - https://github.com/fmzquant/strategies/blob/master/BB-Keltner-Squeeze%25E4%25BA%25A4%25E6%2598%2593%25E7%25AD%2596%25E7%2595%25A5BB-Keltner-Squeeze-Trading-Strategy.md

  ---
  Alexander Elder's Triple Screen Trading System

  Three-Screen Concept:
  - Screen 1 (Weekly): Trend identification using MACD Histogram
  - Screen 2 (Daily): Entry timing using Force Index (2-period EMA)
  - Screen 3 (Intraday): Entry trigger using price action breakouts

  Screen 1 - Trend Identification:
  - Timeframe: Weekly chart (or 5× trading timeframe)
  - Indicator: MACD Histogram
  - Bullish: Histogram slope is UP (rising bars)
  - Bearish: Histogram slope is DOWN (falling bars)
  - Rule: Only trade in direction of weekly trend

  Screen 2 - Entry Timing:
  - Timeframe: Daily chart
  - Indicator: Force Index (2-period EMA)
  - For LONGS: Wait for Force Index to turn negative (pullback below zero)
  - For SHORTS: Wait for Force Index to turn positive (pullback above zero)

  Screen 3 - Entry Trigger:
  - Place buy stop above previous day's high (for longs)
  - Place sell stop below previous day's low (for shorts)
  - Enter when price triggers the stop order

  Stop Loss & Exit Rules:
  - Stop loss: Below/above recent swing low/high
  - Profit target: 2:1 or 3:1 reward-to-risk ratio
  - Exit when weekly MACD Histogram changes slope direction
  - Use trailing stops to lock in profits

  Sources:
  - https://www.investopedia.com/articles/trading/03/040903.asp
  - https://www.quantifiedstrategies.com/alexander-elder-triple-screen-strategy/
  - https://tradingstrategyguides.com/alexander-elder-trading-strategy-the-triple-screen/
  - https://www.mql5.com/en/blogs/post/731016

  ---
  SuperTrend & Half Trend Indicators

  SuperTrend Indicator:
  - ATR-based trend following indicator
  - Formula: Basic Band = (High + Low) / 2
  - SuperTrend = Basic Band ± (Multiplier × ATR)
  - Default settings: Period = 10, Multiplier = 3

  Performance Data:
  - Win Rate: 46-67% (varies by market conditions)
  - Best for: Trending markets, performs poorly in ranging markets
  - Profit per Trade: ~11% (QuantifiedStrategies backtest)

  Entry Rules:
  - Buy when price closes above SuperTrend line (color change to green)
  - Sell when price closes below SuperTrend line (color change to red)
  - Acts as both entry signal and trailing stop

  Half Trend Indicator:
  - Modified version designed to reduce lag
  - Performance: 48.75% return, -144.35% max DD, 51.08% win rate
  - Provides trend direction and potential reversal points
  - Best combined with RSI or ADX for confirmation

  Limitations:
  - Both perform poorly in sideways/ranging markets
  - Require trending markets for optimal performance
  - Work best when combined with other indicators

  Sources:
  - https://www.quantifiedstrategies.com/supertrend-indicator-trading-strategy/
  - https://www.litefinance.org/blog/for-beginners/best-technical-indicators/half-trend-indicator/
  - https://www.metatradermt4.com/metatrader-4/half-trend-indicator-mt4/
  - https://www.howtotrade.com/indicators/half-trend/

  ---
  VWAP Bands (Volume Weighted Average Price with Standard Deviation Bands)

  What is VWAP:
  VWAP = Cumulative(Price × Volume) / Cumulative Volume
  - Institutional benchmark for execution quality
  - Acts as dynamic support/resistance intraday

  VWAP Bands Calculation:
  - Upper Band = VWAP + (Multiplier × Standard Deviation)
  - Lower Band = VWAP - (Multiplier × Standard Deviation)
  - Typical multiplier: 1.5 to 3 standard deviations

  Trading Strategy:

  Long Entry:
  - Price pulls back to VWAP from above
  - Shows rejection at VWAP or lower band
  - Enter on confirmation candle with volume

  Short Entry:
  - Price rallies to VWAP from below
  - Shows rejection at VWAP or upper band
  - Enter on confirmation candle with volume

  Mean Reversion Strategy:
  - Buy when price touches lower VWAP band
  - Sell when price touches upper VWAP band
  - Target: Return to VWAP (mean)
  - Works best in range-bound intraday markets

  Best Used For:
  - Day trading (intraday only)
  - Identifying fair value and institutional levels
  - Support/resistance from volume-weighted price

  Sources:
  - https://help.trendspider.com/kb/indicators/vwap-with-st-dot-dev-bands
  - https://www.tradingview.com/script/fIGZoLqS/Multi-Day-Rolling-VWAP-with-Deviation-Bands/
  - https://www.investopedia.com/ask/answers/031115/what-common-strategy-traders-implement-when-using-volume-weighted-average-price-vwap.asp

  ---
  Machine Learning & AI in Trading

  Neural Network Approaches:
  - Feedforward Neural Networks for price prediction
  - LSTM (Long Short-Term Memory) for time series
  - CNN (Convolutional Neural Networks) for pattern recognition
  - Transformer models for sequence prediction

  Tree-Based Models:
  - Random Forest: Ensemble of decision trees, reduces overfitting
  - XGBoost: Gradient boosting, highly competitive in Kaggle competitions
  - LightGBM: Faster training, memory efficient
  - CatBoost: Handles categorical features well

  Alpha Generation Techniques:
  - Feature engineering from market data
  - Cross-validation and walk-forward testing
  - Ensemble methods combining multiple models
  - Reinforcement learning for strategy optimization

  Key Resources:
  - GitHub: stefan-jansen/machine-learning-for-trading (comprehensive guide)
  - Coursera: Machine Learning for Trading Specialization
  - Quantra: AI in Trading Advanced Track
  - Academic papers on deep learning for algorithmic trading

  Best Practices:
  - Separate train/validation/test sets by time (no future data leakage)
  - Use walk-forward validation instead of random cross-validation
  - Feature selection to prevent overfitting
  - Regularization techniques (L1/L2, dropout, early stopping)

  Sources:
  - https://github.com/stefan-jansen/machine-learning-for-trading
  - https://trendspider.com/learning-center/artificial-intelligence-machine-learning-and-neural-networks-in-trading-an-overview/
  - https://www.coursera.org/specializations/machine-learning-trading
  - https://www.sciencedirect.com/science/article/pii/S2590005625000177
  - https://arxiv.org/abs/1712.09592

  ---
  Market Microstructure & Order Flow Analysis

  Order Flow Imbalance (OFI):
  - Measures difference between buy and sell order volume
  - OFI = (Buy Volume - Sell Volume) / Total Volume
  - Predicts short-term price movements
  - High OFI = Strong buying pressure (price likely to rise)

  VPIN (Volume-Synchronized Probability of Informed Trading):
  - Measures order flow toxicity
  - High VPIN = High risk of informed traders (adverse selection)
  - Used to adjust trading activity during toxic periods

  Limit Order Book (LOB) Metrics:
  - Depth analysis at multiple price levels
  - Bid/ask spread changes
  - Order cancellation rates
  - Queue position dynamics

  Trading Applications:
  - Detect order flow imbalance for entry signals
  - Use VPIN to avoid toxic periods
  - Trade with the "smart money" (informed traders)
  - Optimize execution using LOB depth

  Key Insight:
  The linear relation between order flow imbalance and price changes:
  - Slope is inversely proportional to market depth
  - Shallow markets: Large price impact from order imbalance
  - Deep markets: Small price impact from order imbalance

  Sources:
  - https://hftbacktest.readthedocs.io/en/latest/tutorials/Market%2520Making%2520with%2520Alpha%2520-%2520Order%2520Book%2520Imbalance.html
  - https://www.emergentmind.com/topics/order-flow-imbalance
  - https://www.bookmap.com/blog/how-order-flow-imbalance-can-boost-your-trading-success
  - https://papers.ssrn.com/sol3/papers.cfm?abstract_id=1712822
  - https://www.federalreserve.gov/econres/notes/feds-notes/order-flow-imbalances-and-amplification-of-price-movements-evidence-from-u-s-treasury-markets-20251103.html

  ---
  Regime Switching & Hidden Markov Models (HMM)

  What are Market Regimes:
  - Bull Market: Rising prices, low volatility
  - Bear Market: Falling prices, high volatility
  - Neutral/Stagnant: Sideways movement
  - Regimes are not directly observable (hidden)

  Hidden Markov Models for Trading:
  - Uses statistical inference to detect market states
  - Transition probabilities between regimes
  - Emission probabilities (how price behaves in each regime)
  - Forward algorithm for filtering, Viterbi for most likely state sequence

  Trading Strategy Based on Regimes:
  - Bull regime: Use trend-following strategies (momentum)
  - Bear regime: Use defensive strategies (cash, short positions)
  - Neutral regime: Use mean-reversion strategies
  - Switch strategies when regime probability > threshold

  Common HMM Configurations:
  - 2-state: Bull/Bear
  - 3-state: Bull/Bear/Neutral
  - 4-state: Strong Bull/Mild Bull/Mild Bear/Strong Bear

  Performance Benefits:
  - Avoids drawdowns by switching to defensive in bear markets
  - Captures upside by using trend-following in bull markets
  - Outperforms static strategies in most studies

  Sources:
  - https://papers.ssrn.com/sol3/Delivery.cfm/5206522.pdf?abstractid=5206522
  - https://www.mdpi.com/1911-8074/13/12/311
  - https://questdb.com/glossary/market-regime-detection-using-hidden-markov-models/
  - https://www.quantconnect.com/docs/v2/research-environment/applying-research/hidden-markov-models
  - https://arxiv.org/pdf/2406.02297

  ---
  Time Series vs Cross-Sectional Momentum

  Time Series Momentum (TSMOM):
  - Evaluates each asset against its own past performance
  - Go long if past return > 0, short if < 0
  - More robust across different lookback/holding periods
  - Works better in stable and prosperous environments

  Cross-Sectional Momentum (XSMOM):
  - Ranks assets relative to each other
  - Buy winners, sell losers based on relative performance
  - More effective for shorter lookback periods
  - Benefits from relative stock ranking

  Dual Momentum:
  - Combines both time-series and cross-sectional
  - Outperforms standalone strategies
  - Significantly higher risk-adjusted returns
  - Absolute momentum + relative momentum

  Momentum Crashes:
  - Severe drawdowns affecting momentum strategies
  - Occur during market recoveries after sharp declines
  - Can be mitigated with volatility scaling
  - Risk management: Position sizing based on volatility

  Key Performance Insight:
  - TSMOM: Maintains significant payoffs for longer holding periods
  - XSMOM: Better for shorter holding periods
  - Relationship: Beta ≈ 0.66 between TSMOM and XSMOM
  - Hybrid approaches can enhance performance

  Sources:
  - https://www.tandfonline.com/doi/full/10.1080/23322039.2017.1339772
  - https://thesis.eur.nl/pub/60109/Bachelor_Scriptie_Final_Version_Fayyaz_Sheikh_484438.pdf
  - https://papers.ssrn.com/sol3/Delivery.cfm/SSRN_ID2089463_code753937.pdf?abstractid=2089463
  - http://www.diva-portal.org/smash/get/diva2:1827867/FULLTEXT01.pdf
  - https://www.nber.org/system/files/working_papers/w18169/revisions/w18169.rev1.pdf

  ---
  Higher Moments: Skewness & Kurtosis in Trading

  Beyond Volatility (Variance):
  - Skewness: Asymmetry of return distribution
  - Kurtosis: "Tailedness" or fat-tail risk
  - These higher moments capture tail risk better than variance alone

  Skewness Trading:
  - Negative skew: Higher crash risk (downside tail)
  - Positive skew: More upside potential
  - Skewness risk premium: Compensation for bearing skewness risk
  - Trading: Buy assets with positive skew, sell/avoid negative skew

  Kurtosis Trading:
  - High kurtosis: Fat tails, extreme events more likely
  - Low kurtosis: More normally distributed
  - Tail risk hedging using out-of-the-money options
  - Adjust position sizes based on kurtosis

  Portfolio Optimization with Higher Moments:
  - Mean-Variance-Skewness-Kurtosis framework
  - Incorporates higher moments beyond traditional variance
  - Better captures tail risk and asymmetry
  - Improves risk-adjusted returns

  Option-Implied Moments:
  - Risk-neutral skewness and kurtosis from options
  - Predicts future returns and volatility
  - Trading strategies based on implied moments
  - Greeks extension: Vega (volatility), Vanna (volatility sensitivity to spot), Vomma (volatility of volatility)

  Sources:
  - https://www.tandfonline.com/doi/full/10.1080/1331677X.2017.1340182
  - https://arxiv.org/pdf/2409.13516
  - https://papers.ssrn.com/sol3/Delivery.cfm/SSRN_ID2019526_code83248.pdf?abstractid=1732640
  - https://blog.harbourfronts.com/2025/06/16/using-skewness-and-kurtosis-to-enhance-trading-and-risk-management/
  - https://www.investopedia.com/terms/k/kurtosis.asp

  ---
  Seasonality & Calendar Effects

  What is Seasonality:
  - Price undergoes similar, predictable changes around the same period each calendar year
  - Examples: "Sell in May and go away," January effect, holiday effects

  Types of Seasonality:
  - Annual: Patterns repeating every year
  - Monthly: Day-of-month effects
  - Weekly: Day-of-week effects
  - Intraday: Time-of-day patterns

  Trading Seasonality:
  - Combine historical data analysis with trading rules
  - Identify seasonal patterns using statistical tests
  - Enter/exit based on calendar dates
  - Use multiple seasonal cycles simultaneously

  Holiday Effects:
  - Pre-holiday rally (positive returns before holidays)
  - Post-holiday decline
  - Effects vary by market and region
  - Require special modeling due to moving dates

  Trading Day Effects:
  - Activities varying by day of week create patterns
  - Monday effects, Friday effects
  - Options expiration days
  - Triple witching

  Implementation Considerations:
  - Adjust for trading day effects
  - Remove seasonal component to see true trend
  - Multiple seasonal patterns exist simultaneously
  - High-frequency data reveals complex seasonality

  Sources:
  - https://blog.quantinsti.com/seasonality-trading/
  - https://otexts.com/fpp2/tspatterns.html
  - https://www.census.gov/content/dam/Census/library/working-papers/2018/adrm/rrs2018-01.pdf
  - https://www.ons.gov.uk/economy/economicoutputandproductivity/output/articles/economicactivityandsocialchangeintheukrealtimeindicatorsseasonaladjustment/2025-06-25

  ---
  SUMMARY: Missing Indicators to Consider Adding

  Based on comprehensive research, consider adding these features to your system:

  Volatility-Based:
  - Keltner Channels (ATR-based, alternative to BB)
  - SuperTrend (ATR bands, clear trend signals)
  - Half Trend (reduced lag trend indicator)

  Volume-Price Analysis:
  - VWAP Bands (institutional levels)
  - VPIN (order flow toxicity detection)
  - Order Flow Imbalance (real-time buying/selling pressure)

  Multi-Timeframe:
  - Triple Screen approach (Elder's system)
  - Daily-4H-1H confluence analysis

  Market Regime Detection:
  - Hidden Markov Models (bull/bear/neutral states)
  - Volatility regime switching (your system has ATR14_SMA50, good!)

  Advanced Concepts:
  - Time series momentum (vs cross-sectional)
  - Skewness and kurtosis (tail risk metrics)
  - Seasonality filters (calendar effects)

  Machine Learning Ready:
  - Feature selection (reduce from 60+ to 10-15)
  - Tree-based models (XGBoost, Random Forest)
  - Neural network architectures (LSTM, Transformer)

  Key Recommendation:
  Your system already has excellent features (BOS, FVG, Sweep, Imbalance, VolZ). Focus on:
  1. Reducing feature count to prevent overfitting
  2. Adding Keltner Channels for volatility squeeze detection
  3. Implementing VWAP for intraday levels (if trading 5m timeframe)
  4. Using HMM for regime detection (you have ATR14_SMA50, extend this!)
  5. Testing time-series momentum (your ROC features enable this)

  ---
  PART 14: ADVANCED QUANTITATIVE STRATEGIES

  Hawkes Processes & Order Clustering

  What are Hawkes Processes:
  - Self-exciting point processes for modeling clustered events
  - Arrival intensity depends on recent arrivals of the process itself
  - Captures temporal clustering in order flows

  Applications in Trading:
  - Order arrival modeling (trades cluster in time)
  - Market impact analysis
  - High-frequency signal generation
  - Order flow intensity forecasting

  Key Insight for Alpha Generation:
  - Order arrivals are NOT independent (violates Poisson assumption)
  - Information cascades cause clustering
  - Algorithmic trading amplifies clustering
  - Can identify when order flow is unusually intense (potential informed trading)

  Bivariate Marked Hawkes Process:
  - Models aggressive market order arrivals
  - Each trade increases probability of subsequent trades
  - Can forecast short-term order arrival rates
  - Useful for optimal execution timing

  Sources:
  - https://arxiv.org/pdf/2503.14814
  - https://www.sciencedirect.com/science/article/abs/pii/S1544612316301490
  - https://pmc.ncbi.nlm.nih.gov/articles/PMC6953867/
  - https://jheusser.github.io/2013/09/08/hawkes.html
  - https://www.maths.ox.ac.uk/system/files/attachments/Hawkes%2520Process-Driven%2520Models%2520for%2520Limit%2520Order%2520Book%2520Dynamics_0.pdf

  ---
  Lead-Lag Effects & Cross-Asset Prediction

  What are Lead-Lag Effects:
  - Some assets "lead" (move first)
  - Other assets "lag" (follow the leader)
  - Creates predictable relationships for trading

  Methodology:
  - Granger Causality Tests: Establishes causal relationships
  - Cross-Correlation Analysis: Identifies lagged relationships
  - Vector Autoregressive (VAR) Models: Multi-asset time series analysis
  - Temporal Graph Learning: Advanced ML for lead-lag detection

  Trading Applications:
  - Trade the follower when leader moves
  - Pairs trading with lead-lag relationships
  - Cross-asset hedging
  - Signal generation from leader assets

  Key Findings:
  - High-frequency traders can profit from lead-lag relationships
  - Profitability exists even accounting for trading costs and latency
  - Detecting stock pairs with significant lead-lag benefits investors
  - Followers' price movements mimic leaders

  Practical Implementation:
  1. Identify significant lead-lag pairs (Granger causality)
  2. Calculate optimal lag period (could be seconds to days)
  3. Generate signals from leader asset movement
  4. Trade follower asset with appropriate delay

  Sources:
  - https://www.tandfonline.com/doi/abs/10.1080/07350015.2019.1697699
  - https://arxiv.org/html/2506.19255v1
  - https://www.sciencedirect.com/science/article/abs/pii/S0927538X25003397
  - https://papers.ssrn.com/sol3/Delivery.cfm/c920b815-533e-4d93-b303-e4b62287a054-MECA.pdf?abstractid=5043260
  - https://link.springer.com/article/10.1186/s40854-022-00356-3

  ---
  Market Neutral & Beta Neutral Strategies

  Market Neutral Concepts:
  - Strategies designed to profit regardless of overall market direction
  - Eliminates market risk (beta ≈ 0)
  - Focuses on stock-specific alpha

  Beta Neutral:
  - Portfolio construction to eliminate systematic risk
  - Hedge ratio adjustment based on beta to market
  - Long/short positions weighted by beta
  - Profit from stock selection, not market movement

  Pairs Trading (Market Neutral):
  - Long undervalued, short overvalued in a pair
  - Cointegration-based: Statistical arbitrage
  - Mean reversion: Profits from price convergence
  - Market neutral: Profits independent of market direction

  Statistical Arbitrage:
  - Quantitative models using mean reversion
  - Large security portfolios over short periods
  - Exploits pricing inefficiencies
  - Short holding periods (hours to days)

  Key Advantage:
  - Consistent profits in both bull and bear markets
  - Lower volatility than directional strategies
  - Focus on alpha generation, not beta exposure

  Sources:
  - https://bookmap.com/blog/pairs-trading-a-deep-dive-into-this-market-neutral-strategy
  - https://analystprep.com/study-notes/cfa-level-iii/statistical-arbitrage-and-microstructure/
  - https://www.investopedia.com/terms/s/statisticalarbitrage.asp
  - https://www.econstor.eu/bitstream/10419/116783/1/833997289.pdf
  - https://www.aima.org/asset/BF985C5A-7BC5-471C-BC2549C71EC504AE/

  ---
  Harmonic Patterns & Fibonacci Trading

  Harmonic Patterns Definition:
  - Mathematical patterns based on Fibonacci numbers and Golden Ratio
  - Geometric price formations using Fibonacci ratios
  - Identify potential reversal zones (PRZ)

  Key Fibonacci Ratios:
  ┌────────────┬──────────────────┬─────────────────────────────────┐
  │ Ratio      │ Value             │ Usage                              │
  ├────────────┼──────────────────┼─────────────────────────────────┤
  │ Golden     │ 0.618 / 1.618      │ Primary retracement/extension      │
  ├────────────┼──────────────────┼─────────────────────────────────┤
  │ 0.786      │ √0.618 / 2        │ Plot point B in harmonic patterns │
  ├────────────┼──────────────────┼─────────────────────────────────┤
  │ 0.382      │ 1 - 0.618         │ Secondary retracement            │
  ├────────────┼──────────────────┼─────────────────────────────────┤
  │ 1.272      │ 1.618 / 1.272      │ Extension level                  │
  └────────────┴──────────────────┴─────────────────────────────────┘

  Primary Harmonic Patterns:
  1. **Gartley Pattern**: Bullish/Bearish, AB=CD, XA retracement to D
  2. **Butterfly Pattern**: Perfect symmetry, extended PRZ
  3. **Bat Pattern**: 0.886 XA retracement, precise PRZ
  4. **Crab Pattern**: Most harmonic, 1.618 XA projection
  5. **Deep Crab**: Deepest retracement (0.886), 2.618 XA
  6. **Cypher Pattern**: 0.786 XA and YC retracements
  7. **ABCD**: Simple pattern, 0.618 or 0.786 retracements

  Trading Harmonic Patterns:
  - Identify the pattern (XABCD points)
  - Calculate Potential Reversal Zone (PRZ)
  - Enter at PRZ with confluence (Fibonacci + structure)
  - Stop loss beyond pattern invalidation point
  - Target: Complete the pattern (D point)

  Key Validation:
  - AB=CD symmetry
  - Specific Fibonacci retracement levels
  - Pattern completion at PRZ
  - Time zone alignment (optional)

  Sources:
  - https://www.ig.com/en/trading-strategies/top-7-harmonic-patterns-every-trader-should-know-210608
  - https://www.investopedia.com/articles/forex/11/harmonic-patterns-in-the-currency-markets.asp
  - https://smartrisk.net/top-8-harmonic-patterns-every-trader-should-know/
  - https://algotrading-investment.com/2020/06/04/fibonacci-retracement-and-expansion-patterns/
  - https://harmonictrader.com/harmonic-patterns/butterfly-pattern/
  - https://trendspider.com/learning-center/harmonic-patterns/

  ---
  Delta Neutral Options Trading & Gamma Scalping

  Option Greeks Overview:
  ┌────────┬─────────────────────┬─────────────────────────────────────┐
  │ Greek   │ Measures              │ Impact                                │
  ├────────┼─────────────────────┼─────────────────────────────────────┤
  │ Delta  │ Price sensitivity     │ Directional exposure, hedge ratio    │
  ├────────┼─────────────────────┼─────────────────────────────────────┤
  │ Gamma  │ Delta sensitivity     │ Rate of delta change, stability     │
  ├────────┼─────────────────────┼─────────────────────────────────────┤
  │ Theta  │ Time decay            │ Daily loss from option erosion      │
  ├────────┼─────────────────────┼─────────────────────────────────────┤
  │ Vega   │ Volatility sensitivity │ Exposure to implied volatility changes│
  └────────┴─────────────────────┴─────────────────────────────────────┘

  Delta Neutral Strategy:
  - Portfolio delta = 0 (no directional exposure)
  - Hedge by adjusting position to offset delta
  - Profit from gamma scalping (trading volatility)
  - Income from theta decay (time premium)

  Gamma Scalping:
  - Dynamic hedging to maintain delta neutrality
  - Buy low sell high as price oscillates
  - Profit from realized volatility exceeding implied volatility
  - Requires continuous monitoring and adjustment

  Trade-Off: Gamma vs Theta
  - Long gamma: Benefit from large moves, pay theta daily
  - Short gamma: Collect theta, risk from large moves
  - Gamma scalping: Capture intraday volatility, pay theta at night

  Risk Considerations:
  - Gamma risk: Delta changes rapidly (especially near expiration)
  - Vega risk: Volatility spikes can cause large losses
  - Pin risk: Assignment risk for short options
  - Liquidity risk: Bid-ask spreads in options

  Sources:
  - https://www.investopedia.com/trading/getting-to-know-the-greeks/
  - https://profitmart.in/blog/gamma-scalping-and-hedging/
  - https://kth.diva-portal.org/smash/get/diva2:1905703/FULLTEXT01.pdf
  - https://www.linkedin.com/pulse/gamma-hidden-enemy-delta-neutral-strategies-0dte-joaquin-bejar-garcia-ovlsf
  - https://papers.ssrn.com/sol3/Delivery.cfm/5285239.pdf?abstractid=5285239

  ---
  Kelly Criterion for Position Sizing

  What is Kelly Criterion:
  - Formula for sizing bets to maximize long-term growth
  - Maximizes expected logarithm of wealth
  - Balances growth and risk of ruin

  Kelly Formula:
  f* = (bp - q) / b

  Where:
  - f* = Fraction of capital to wager
  - b = Odds received on the wager (b to 1)
  - p = Probability of winning
  - q = Probability of losing (1 - p)

  Trading Version:
  f* = (W × R - L) / R

  Where:
  - f* = Optimal position size (% of capital)
  - W = Win rate (as decimal, e.g., 0.55)
  - R = Average win / Average loss (Reward:Risk ratio)
  - L = Loss rate (1 - W)

  Key Insight:
  - Overbetting → Risk of ruin
  - Underbetting → Suboptimal growth
  - Half Kelly: Conservative approach, half the Kelly fraction
  - Multiple uncorrelated strategies: Can sum Kelly fractions

  Example Calculation:
  - Win rate: 55% (W = 0.55)
  - Reward:Risk: 1.5:1 (R = 1.5)
  - f* = (0.55 × 1.5 - 0.45) / 1.5 = 0.375
  - Bet 37.5% of capital per trade
  - Half Kelly: 18.75% for conservative approach

  Risk Management:
  - Kelly assumes known probabilities (markets are uncertain)
  - Use half-Kelly or quarter-Kelly for safety
  - Adjust for estimation error (your edge may be less than you think)
  - Correlated strategies: Reduce total position size

  Sources:
  - https://en.wikipedia.org/wiki/Kelly_criterion
  - https://www.investopedia.com/articles/trading/04/091504.asp
  - https://medium.com/@jatinnavani/the-kelly-criterion-and-its-application-to-portfolio-management-3490209df259
  - https://arxiv.org/pdf/1710.00431
  - https://blog.quantinsti.com/risk-constrained-kelly-criterion/

  ---
  Correlation Breakdown & Contagion Risk

  Correlation Breakdown:
  - In crises, correlations tend to +1 (everything falls together)
  - Traditional diversification fails during market stress
  - Safe havens may NOT provide protection in extreme events

  Key Research Findings:
  - Gold failed as safe haven during extreme contagion
  - US markets exert largest influence on global contagion
  - Correlation networks don't necessarily cause transmission chains
  - Cross-border linkages show surprising similarity in crisis behavior

  Contagion Channels:
  - Direct channels: Trade, financial linkages
  - Indirect channels: Investor behavior, sentiment
  - Flight to quality: Sometimes occurs, sometimes fails
  - Volatility transmission: Spreads across markets

  Systemic Risk Indicators:
  - Correlation breakdown risk: Measure of instability
  - Network centrality: Systemically important institutions
  - Cross-asset contagion: Dynamic spillover effects
  - VPIN: Order flow toxicity (informed trading risk)

  Portfolio Protection Strategies:
  1. Tail risk hedging (out-of-the-money puts)
  2. Volatility targeting (reduce size when VIX high)
  3. Dynamic correlation monitoring
  4. Safe haven rotation (with caution)
  5. Cash buffers for extreme events

  Key Insight:
  Traditional diversification can fail when you need it most.
  Consider:
  - Cash as the only true safe haven
  - Dynamic hedging strategies
  - Reducing exposure during high contagion periods

  Sources:
  - https://www.sciencedirect.com/science/article/abs/pii/S0927538X25001039
  - https://papers.ssrn.com/sol3/Delivery.cfm/SSRN_ID1699231_code642039.pdf?abstractid=1691151
  - https://www.ecb.europa.eu/pub/pdf/scpwps/ecbwp071.pdf
  - http://sf.ruc.edu.cn/docs/2024-11/0243f415723d496da8c104f1b1b85852.pdf
  - https://arxiv.org/pdf/2012.12702
  - https://www.sciencedirect.com/science/article/pii/S1062940825001512

  ---
  PART 15: COMPREHENSIVE INDICATOR CHECKLIST

  Complete List of Indicators & Strategies Researched:

  Trend Following (15+):
  ✓ EMA (multiple periods) - You have
  ✓ HMA (multiple periods) - You have
  ✓ Kijun-sen (Ichimoku) - You have
  ✓ ROC/Slope - You have
  ✓ SuperTrend - Consider adding
  ✓ Half Trend - Consider adding
  ✓ Golden/Death Cross (50/200 MA)
  ✓ Donchian Channels (20/55)
  ✓ Parabolic SAR
  ✓ Triple Screen (Elder)
  ✓ Time Series Momentum (12-month)
  ✓ Dual Momentum (relative + absolute)

  Momentum & Oscillators (20+):
  ✓ RSI (multiple periods) - You have
  ✓ MACD - You have
  ✓ Stochastic - Consider adding
  ✓ Williams %R - Consider adding
  ✓ CCI - Consider adding
  ✓ MFI - You have
  ✓ Force Index - Consider adding
  ✓ Money Flow Index
  ✓ Elder's Force Index
  ✓ Divergence patterns (all oscillators)

  Volatility-Based (10+):
  ✓ ATR (multiple periods) - You have
  ✓ Bollinger Bands - You have
  ✓ Keltner Channels - Consider adding (BB-KC squeeze)
  ✓ VOLRET_20 - You have
  ✓ SuperTrend - Consider adding
  ✓ Standard deviation channels
  ✓ ATR channels
  ✓ Historical volatility vs implied volatility

  Volume-Based (8+):
  ✓ OBV - You have
  ✓ VolSMA/VolEMA - You have
  ✓ VolZ - You have
  ✓ VolPerTrade - You have
  ✓ BuyRatio - You have
  ✓ Imbalance - You have
  ✓ VWAP - Consider adding for intraday
  ✓ VPIN (order flow toxicity) - Consider adding
  ✓ Order Flow Imbalance - Already captured (Imbalance)

  Market Structure (Advanced):
  ✓ SwingHigh/SwingLow - You have
  ✓ HH/LL (multiple periods) - You have
  ✓ BOS (Break of Structure) - You have
  ✓ FVG (Fair Value Gaps) - You have
  ✓ Sweep - You have
  ✓ Displacement - You have
  ✓ Order flow (market microstructure)
  ✓ Lead-lag relationships
  ✓ Hawkes processes (order clustering)

  Multi-Timeframe:
  ✓ Daily-4H-1H confluence
  ✓ Triple Screen (Elder)
  ✓ MTF alignment

  Regime Detection:
  ✓ ATR14_SMA50 (volatility filter) - You have
  ✓ Hidden Markov Models - Consider adding
  ✓ Market states (bull/bear/neutral)
  ✓ Volatility regimes

  Statistical/Quantitative:
  ✓ Cointegration (pairs trading)
  ✓ Z-score normalization (pairs)
  ✓ Correlation analysis
  ✓ Granger causality
  ✓ Stationarity testing
  ✓ Half-life of mean reversion
  ✓ Kurtosis & Skewness (tail risk)
  ✓ Seasonality effects
  ✓ Calendar anomalies

  Risk Management:
  ✓ Kelly Criterion (position sizing)
  ✓ ATR-based stops
  ✓ Percentage stops
  ✓ Trailing stops
  ✓ Portfolio optimization
  ✓ Tail risk hedging
  ✓ Correlation breakdown risk

  Machine Learning:
  ✓ Random Forest
  ✓ XGBoost
  ✓ LSTM networks
  ✓ Transformer models
  ✓ Feature selection (LASSO, Ridge)
  ✓ Walk-forward validation
  ✓ Ensemble methods

  Options Strategies (if applicable):
  ✓ Delta neutral
  ✓ Gamma scalping
  ✓ Theta decay
  ✓ Vega hedging
  ✓ Iron Condor
  ✓ Butterfly spread
  ✓ Calendar spread

  FINAL SUMMARY FOR YOUR SYSTEM:

  What You Already Have (Excellent):
  - Trend: EMA, HMA, Kijun26, ROC, SLOPE
  - Momentum: RSI, MACD, MFI, ROC5
  - Volatility: ATR, BB, VOLRET, RangeWidth
  - Volume: OBV, VolSMA/VolEMA, VolZ, BuyRatio, Imbalance, VolPerTrade
  - Structure: SwingHigh/Low, BOS, FVG, Sweep, Displacement, HH/LL
  - Price Action: Body, BodyPct, WickUpPct, WickDownPct, ClosePos
  - Regime Filter: ATR14_SMA50 ✓

  What to Consider Adding:
  HIGH PRIORITY:
  - Keltner Channels (for BB-KC squeeze detection)
  - SuperTrend (simple trend following)
  - Stochastic (oversold/overbought)
  - Williams %R (similar to RSI but different calculation)
  - Hidden Markov Models (extend your regime filter)

  MEDIUM PRIORITY:
  - VWAP (if trading intraday)
  - Triple Screen concepts (multi-timeframe)
  - Seasonality filters (day-of-week, month effects)
  - Lead-lag detection (if trading multiple assets)

  LOW PRIORITY:
  - Harmonic patterns (complex, discretionary)
  - Options Greeks (requires options data)
  - Hawkes processes (requires tick data)
  - Kelly Criterion (for position sizing, not entry signals)

  KEY RECOMMENDATIONS:
  1. Add Keltner Channels for volatility squeeze breakout detection
  2. Implement Hidden Markov Model to extend ATR14_SMA50 regime filter
  3. Test Triple Screen approach using your existing indicators
  4. Reduce feature count from 60+ to 10-15 most predictive
  5. Focus on walk-forward validation and out-of-sample testing
  6. Target: Beat 12% CAGR with <30% max drawdown
  7. Paper trade for 6-12 months before live deployment

  Your system is already sophisticated with advanced features (BOS, FVG, Sweep, Imbalance, VolZ).
  The key is SIMPLIFICATION and ROBUSTNESS, not adding more complexity.
